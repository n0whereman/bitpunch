TODO: v skratke, co chceme dosiahnut...

In today's world, cryptography has found its place in many areas of ordinary living and it is important to understand its limits. We have learnt the meaning of the word privacy, especially on the internet. We distribute private information on web sites, we save confidential information on our computers, we pay through internet banking believing and expecting that all of this is adequately defended. Quantum computer may be a breakthrough in current cryptography, regarding its enormous power.

On of the best candidates for post-quantum cryptography comes from times back to the beginning of modern cryptography. McElice has proposed cryptosystem based on error code problem, which was proven to be NP hard. In recent years, dedicated to implementation of cryptographic library called Bitpunch. The project was a part of programme called: "Secure implementation of post-quantum cryptography", NATO Science for Peace and Security Programme Project Number: 984520. 

\indent In the first section, we discuss needed preliminaries, that are necessary to understand the McEliece cryptosystem.
The further section is dedicated to basic analysis of Bitpunch and its CCA security. We also provide an overview of missing key encapsulation management. 
The third section presents our contribution to this project. The fifth section shows basic conducted test of the implementation. In the end of this thesis, we conclude accomplished results.

  
